\documentclass[12pt]{article}
\usepackage[utf8]{inputenc}
\usepackage{geometry}
\geometry{
	a4paper,
	left=20mm,
	right=20mm,
	top=25mm,
	bottom=20mm
}
\usepackage{amsmath}
\usepackage{amsfonts}
\usepackage{amssymb}
\usepackage{hyperref}
\hypersetup{
    colorlinks=true,
    linkcolor=black,
    filecolor=magenta,      
    urlcolor=cyan,
    pdfpagemode=FullScreen,
    }
\renewcommand{\theenumi}{\roman{enumi}}

\newcommand{\sq}{$\square$}
\newcommand{\rmk}{$\surd$}
\newcommand{\N}{\mathbb{N}}
\newcommand{\Q}{\mathbb{Q}}
\newcommand{\R}{\mathbb{R}}
\newcommand{\U}{\mathcal{U}}
\newcommand{\V}{\mathcal{V}}
\newcommand{\A}{\mathcal{A}}
\newcommand{\B}{\mathcal{B}}
\newcommand{\C}{\mathcal{C}}
\newcommand{\LL}{\mathcal{L}}
\newcommand{\Borel}{\mathcal{B}(\mathbb{R})}
\newcommand{\open}{\underset{open}{\subset}}
\newcommand{\closed}{\underset{closed}{\subset}}
\newcommand{\subsp}{\underset{subsp}{\subset}}
\newcommand{\seq}{\underset{seq}{\subset}}
\newcommand{\cl}{\overline}
\newcommand{\diff}{\, \backslash \,}
\newcommand{\exist}{\exists \;}
\newcommand{\forany}{\; \forall \;}
\newcommand{\union}{\,\cup\,}
\newcommand{\intersect}{\,\cap\,}
\newcommand{\homeo}{\underset{Homeo}{\simeq}}
\newcommand{\simple}{\phi=\sum_{k=1}^n \alpha_{k} \chi_{E_k}}




\begin{document}


\begin{titlepage}
	\begin{center}
		\vspace*{5cm}
		\textbf{\Large Undergraduate Real Analysis Facts}
		\\	
		\vspace{1.5cm}
		\textbf{Taeyoung Chang}
		\vfill
		Last Update : \today
		\vspace*{3cm}
		\thispagestyle{empty}
	\end{center}
\end{titlepage}
\tableofcontents
\clearpage
	
\section{Abstract Measure Theory}
\smallskip
\begin{itemize}
    \item[*] $\sigma$-algebra $\A$ : a collection of sets containing the whole set $X$ and closed under taking complement or countable union \& intersection 
    \item[*] $\sigma$-algebra $\sigma(\C)$ generated by $\C$ : the smallest $\sigma$-algebra containing $\C$  
    \item[*] Borel $\sigma$-algebra $\Borel$ : $\sigma$-algebra generated by $\mathcal{T}=\{$all open subsets of $\R\}$
    \item All open sets, closed sets, and countable sets in $\R$ are Borel sets 
    \item Every open set in $\R$ can be expressed as a countable union of disjoint open intervals.
    \item $\Borel$ can be generated by the collections above :
    \begin{enumerate}
        \item All finite open intervals
        \item All finite closed intervals
        \item All finite left-closed half-open intervals
        \item All finite right-closed half-open intervals
        \item All left-unbounded open rays
        \item All right-unbounded open rays
        \item All left-unbounded closed rays
        \item All right-unbounded closed rays
   
    \end{enumerate}
    \item[\rmk] There are many subsets of $\R$ which are not Borel. But there is no easy construction of non Borel sets. We can say that 'natural' sets we encounter are mostly Borel. 
    \item[\rmk] A countable intersections of open sets in $\R$ is called as $G_\delta$ set. A countable unions of closed sets in $\R$ is called as $F_\sigma$ set. $G_\delta$ sets and $F_\sigma$ sets are Borel sets.
    \item[*] Measure $\mu$ : a set function on a $\sigma$-algebra, which is nonnegative and countably additive satisfying $\mu(\phi)=0$
    \item[(Ex)] The Lebesgues measure $m$ on $\R$
    \begin{itemize}
        \item $m(I)=$ length of $I$ for each interval $I\subset \R$
        \item $m(E+x)=m(E) \forany E\in \Borel,\, x\in \R$ \; i.e. $m$ is invariant under translation
    \end{itemize}
    \item[(Ex)] The counting measure
    \begin{itemize}
        \item $(X, \mathcal{P}(X), \mu)$ where $\mu(E)=|E| \forany E\subset X$.
    \end{itemize}
    \item[(Ex)] Dirac measure or Point mass
    \begin{itemize}
        \item Fix $x\in X$. $(X, \mathcal{P}(X), \delta_x)$ where $\delta_x(A)=I(x\in A) \forany A\subset X$
    \end{itemize} 
    \item[(Ex)] Restriction of measure
    \begin{itemize}
        \item $(X, \A, \mu)$ : measure space. $B\in \A$. Then restriction $\A_B=\{A\intersect B : A\in \A\}$ is a $\sigma$-algebra on $B$ and $\mu |_{\A_B}$ is a measure on $(B, \A_B)$. $\mathcal{B}(B)$ is a restriction of Borel sigma field on a Borel set $B$, which is equal to a $\sigma$-field generated by open sets in subspace topology on $B$ 
    \end{itemize}
    \bigskip
    \item Elementary properties of a measure
    \begin{itemize}
        \item $(X, \A, \mu)$ : a measure space.
        \begin{enumerate}
            \item (Monotonicity) : For any $A,B\in \A$, \; if $A\subset B$ then $\mu(A)\leq \mu(B)$
            \item For any $A,B\in \A$ with $\mu(A)<\infty$, \; if $A\subset B$ then $\mu(B \diff A)= \mu(B)-\mu(A)$
            \item (Subadditivity) : For any $\{A_n\}_n \subset \A$, \; $\mu(\bigcup_n A_n)\leq \sum_{n}\mu(A_n)$
            \item (Continuity from below) : For any $\{A_n\}_n \subset \A$, \; if $A_n\subset A_{n+1}\forany n$, then $\mu(\bigcup_n A_n)=\lim_n \mu(A_n)$
            \item (Continuity from above) :  For any $\{A_n\}_n \subset \A$, \; if $A_n\supset A_{n+1}\forany n$ then $\mu(\bigcap_n A_n)=\lim_n \mu(A_n)$ \; provided $\mu(A_1)<\infty$ 
        \end{enumerate}
    \end{itemize}
    \item[*] Finite measure and $\sigma$-finite measure
    \begin{itemize}
        \item If $\mu$ is a measure on $(X,\A)$ satisfying $\mu(X)< \infty $ then $\mu$ is said to be a finite meausre. If $X = \bigcup_n X_n$ and $\mu(X_n)<\infty \; \forall n\in \N$ then $\mu$ is said to be $\sigma$-finite.
    \end{itemize} 
    \item[*] Null set and complete measure
    \begin{itemize}
        \item $(X, \A, \mu)$ : a measure space. $E\in \A$. $E$ is said to be a null set if $\mu(E)=0$. \\$\mu$ is said to be complete if for any null set $E\in \A$, \; $F\subset E \Rightarrow F\in \A$. 
    \end{itemize}
    \item Completion of measure space
    \begin{itemize}
        \item $(X, \A, \mu)$ : a measure space. $\mathcal{N}$ : A collection of null sets. \\Define $\overline{\A}=\{E\union F : E\in \A, F\subset N$ for some $N\in \mathcal{N}\}$. \\Then (a)$\overline{\A}$ is a $\sigma$-algebra. (b) There is a unique measure $\overline{\mu}$ on $\overline{\A}$ extending $\mu$ given by $\overline{\mu}(E\union F)=\mu(E) \forany E\in \A, F\subset N$ for some $N\in \mathcal{N}$. \\(c) $\overline{\mu}$ is complete measure. $(X, \A, \overline{\mu})$ is a completion of $(X, \A, \mu)$ 
    \end{itemize}
    \item[*] Lebesgue measurable sets
    \begin{itemize}
        \item Denote $\LL(\R)=\overline{\Borel}$ for the completion $(\R, \overline{\Borel}, \overline{m})$ of the measure space $(\R,\Borel, m)$ and call it as the $\sigma$-algebra of Lebesgue measurable sets. We still denote $\overline{m}$ by $m$ (abusing the notation) and call it the Lebesgue measure on $\R$. Every $E\in \LL(\R)$ is called as a Lebesgue measurable set.
    \end{itemize}
\end{itemize}
\clearpage

\section{Integration over a General Measure Space}
\smallskip
\subsection{Measurable Functions}
\smallskip
\begin{itemize}
    \item[*] Measurable Functions
    \begin{itemize}
        \item $(X, \A)$ : a measurable space. $f:X\rightarrow \R \union \{\pm \infty\}$. $f$ is said to be $\A$-measurable if $f^{-1}(B)\in \A \forany B\in \Borel$
    \end{itemize}
    \item[(Ex)] Borel measurable functions
    \begin{enumerate}
        \item Every continuous function $f:\R\rightarrow \R$ is $\Borel$ measurable. Every continuous function $f:I\rightarrow \R$ is $\mathcal{B}(I)$ measurable given $I\subset \R$ is an interval.
        \item Every monotone function $f:\R\rightarrow \R$ is $\Borel$ measurable. Every monotone function $f:I\rightarrow \R$ is $\mathcal{B}(I)$ measurable given $I\subset \R$ is an interval.         
    \end{enumerate} 
    \item[(Ex)] Characteristic function 
    \begin{itemize}
        \item For given $A\subset X$, $\chi_A$ defined as $\chi_A(x)=I(x\in A)$  is said to be the characteristic function. $\chi_A$ is a measurable function $\Leftrightarrow A$ is a measurable set. 
    \end{itemize} 
    \item[(Ex)] Simple function
    \begin{itemize}
        \item A real-valued function $\phi$ is called simple if it is measurable and has only a finite number of values. $\phi=\sum_{k=1}^n \alpha_{k} \chi_{E_k}$ for some scalaras $\alpha_k$'s and measurable sets $E_k$'s.
    \end{itemize}
    \item Any linear combination of finite number of simple functions is simple. Also, any finite product of simple functions is simple.
    \item Consturcting measurable functions
    \begin{itemize}
        \item If $f, g : X\rightarrow \R$ are measurable then $f+c, cf, f\pm g$ and $fg$ are measurable.
        \item If $f_n : X\rightarrow \R$, $n\in \N$ are measurable then $\sup f_n, \inf f_n, \max\{f_1, \cdots, f_n\}, \min\{f_1, \cdots, f_n\},\\ \limsup f_n, \liminf f_n, \lim f_n$ are measurable.
    \end{itemize} 
    \item $f:I\rightarrow \R$ is a function with a finite number of discontinuities i.e. $f$ is piecewise continuous. Then $f$ is $\mathcal{B}(I)$-measurable.
    \item $f$ is real-valued function. $f= f^+ - f^-$ and $f^+f^-=0$ and such decomposition is unique in the sense that if $f= f_1-f_2$ for some nonnegative functions $f_1, f_2$ \; s.t. $f_1f_2=0$ then $f_1=f^+$ and $f_2=f^-$. Also $|f|=f^+ + f^-$.
    \item Simple Approximation theorem
    \begin{itemize}
        \item If $f: X\rightarrow \R$ is a measurable then $\exist$ a sequence $\{\phi_n\}$ of simple functions s.t. 
        \begin{enumerate}
            \item $0\leq |\phi_1|\leq |\phi_2|\leq \cdots \leq |\phi_n|\leq \cdots \leq |f|$
            \item $\phi_n \rightarrow f$ pointwisely.
        \end{enumerate}
        Additionally, if $f$ is nonnegative measurable then above ($\romannumeral 1$) is replaced by \\ $0\leq \phi_1\leq \phi_2\leq \cdots \leq \phi_n\leq \cdots \leq f$
    \end{itemize}
\end{itemize}
\smallskip
\subsection{Integration on Nonnegative Functions}
\begin{itemize}
    \item[*] Integration of nonnegative simple functions
    \begin{itemize}
        \item For a nonenegative simple function $\simple$, we define the interal by $$\int_X \phi \, d\mu= \sum_{k=1}^n \alpha_k \mu(E_k)$$. For meaurable set $E\in \A$, we define $\int_E \phi \, d\mu=\int_X \phi \chi_E \, d\mu$
        \item [\rmk] The definition above is well defined   i.e. if a nonnegative simple function $\phi$ is written as $\simple$ and $\phi= \sum_{j=1}^m \beta_j \chi_{F_j}$ then integral is same for both expressions.
        \item Integral might have the value of $\infty$.
    \end{itemize}
    \item Elementary properties of integral of nonnegative simple functions.
    \begin{itemize}
        \item Let $\varphi$ and $\psi$ be nonnegative simple functions
        \begin{enumerate}
            \item If $\alpha, \beta\geq 0$ then $\int (\alpha \varphi+\beta \psi)\, d\mu=\alpha \int \varphi \, d\mu+\beta \int \psi \, d\mu$
            \item If $\varphi\leq \psi$ then $\int \varphi \, d\mu \leq \int \psi \, d\mu$ 
            \item If $\nu : \A \rightarrow [0,\infty]$ is defined by $E\mapsto \int_E \varphi \, d\mu$ then $\nu$ is a measure.
        \end{enumerate}    
    \end{itemize}
    \item[*] Integration of nonnegative measurable functions
    \begin{itemize}
        \item For a nonnegative measurable function $f:X\rightarrow [0,\infty]$, we define the integral by \[\int _X f \, d\mu=\sup\{\int \phi \, d\mu : 0\leq \phi\leq f, simple\}\].  For meaurable set $E\in \A$, we define $\int_E f \, d\mu=\int f \chi_E \, d\mu$
        \item We denote the space of nonnegative measurable functions by $\LL^+$
    \end{itemize}
    \item Elementary properties of integral of nonnegative measurable functions.
    \begin{itemize}
        \item Let $f,g\in \LL^+$ 
        \begin{enumerate}
            \item If $\alpha, \beta\geq 0$ then $\int (\alpha f+\beta g)\, d\mu=\alpha \int f \, d\mu+\beta \int g \, d\mu$
            \item If $f\leq g$ then $\int f \, d\mu \leq \int g \, d\mu$ 
            \item If $\nu : \A \rightarrow [0,\infty]$ is defined by $E\mapsto \int_E f \, d\mu$ then $\nu$ is a measure.
        \end{enumerate}    
    \end{itemize}
    \item Monotone convergence theorem (MCT)
    \begin{itemize}
        \item If $\{f_n\}$ is a sequence in $\LL^+$ with $f_n\leq f_{n+1}\; \forall n$ then we have $$\lim_n \int f_n \, d\mu= \int \lim_n f_n \, d\mu$$ 
    \end{itemize}
    \item[\sq] If $\{f_n\}$ is a sequence in $\LL^+$ then we have $$ \int \sum_{n=1}^\infty f_n \, d\mu= \sum_{n=1}^\infty \int f_n \, d\mu$$
    \item If $f\in \LL^+$ and $E$ is a measure zero set then $\int_E f\, d\mu=0$.
    \item[*] Almost everywhere
    \begin{itemize}
        \item A statement $P(x)$ depending on $x\in X$ is said to hold almost everywhere if the set $\{x\in X : P(x)$ does not hold $\}$ is a subset of a measure zero set.
    \end{itemize} 
    \item[\sq] Suppose $\{f_n\}$ is a sequence in $\LL^+$ and $f\in \LL^+$. If $f_n\nearrow f\; a.e.$ then $\int f \, d\mu= \lim_n \int f_n \, d\mu$
    \item If $f\in \LL^+$ then $\int f \, d\mu=0 \Leftrightarrow f=0 \; a.e.$
    \item Fatou's lemma
    \begin{itemize}
        \item If $\{f_n\}$ is a sequence in $\LL^+$ then $\int \liminf f_n \, d\mu\leq \liminf \int f_n\, d\mu$
        \item If $\{f_n\}$ is a sequence in $\LL^+$ and $f_n\rightarrow f\; a.e.$ then $\int f \, d\mu = \liminf\int f_n \, d\mu$ 
    \end{itemize}
\end{itemize}
\smallskip

\subsection{Integration for the General Case}
\smallskip
\begin{itemize}
    \item[*] Integrability
    \begin{itemize}
        \item For $f\in \LL^+$, $f$ is said to be integrable if $\int f\, d\mu<\infty$
        \item For a measurable function $f$, $f$ is said to be integrable if both $f^+$ and $f^-$ are integrable, or equivalently, $|f|$ is integrable.
        \item We denote the space of integrable functions by $\LL^1$
    \end{itemize} 
    \item If $f\in \LL^+$ is integrable then $f<\infty \; a.e.$ and $\{f>0\}$ is a $\sigma$-finite set.
    \item If $f\in \LL^+$ is integrable then $\forall\; \epsilon>0, \; \exist E$ measurable set s.t. $\mu(E)<\infty$ and \\ $\int f\, d\mu -\varepsilon < \int_E f\, d\mu$
    \item Elementary properties of integral of integrable functions
    \begin{itemize}
        \item Let $f,g\in \LL^1$ 
        \begin{enumerate}
            \item For $\alpha, \beta\in \R$, \;$\alpha f+\beta g\in \LL^1$ and $\int (\alpha f+\beta g)\, d\mu=\alpha \int f \, d\mu+\beta \int g \, d\mu$
            \item If $f=f_1-f_2$ with $f_1, f_2\in \LL^+ \intersect \LL^1$ then $\int f\, d\mu=\int f_1\, d\mu- \int f_2\, d\mu$
            \item If $f\leq g$ then $\int f \, d\mu \leq \int g \, d\mu$ 
            \item $|\int f \, d\mu|\leq \int |f| \, d\mu$
        \end{enumerate}    
    \end{itemize}
    \item Given $f\in \LL^1$, a map defined by $E\mapsto \int_E f\, d\mu$ has a countable additivity.
    \item If $f\in \LL^1$ and $E$ is a measure zero set then $\int_E f\, d\mu=0$
    \item If $f\in \LL^1$ then $f=0 \; a.e \Rightarrow \int f \, d\mu=0$
    \item If $f, g\in \LL^+$ or $f,g\in \LL^1$ then $f\leq g\;\, a.e. \Rightarrow \int f\, d\mu\leq \int g\, d\mu$  \\and $f=g \;\, a.e. \Rightarrow \int f\, d\mu=\int g\, d\mu$
    \item[(Note)] Summary for facts about $\LL^+$ and $\LL^1$
    \\ 
    \begin{tabular}{|p{5.3cm}||p{4.7cm}|p{5.1cm}|}
        \hline
        Statement about functions & Functions in $\LL^+$ & Functions in $\LL^1$ \\
        \hline
        Closed under linear combi.   & Yes (with positive scalars)  & Yes \\
        $f\leq g\;\, a.e.\Rightarrow \int f\, d\mu\leq \int g\, d\mu$ &  Yes  & Yes \\
        $f=g\;\, a.e.\Rightarrow \int f\, d\mu = \int g\, d\mu$ & Yes & Yes \\
        $E\mapsto \int_E f\, d\mu$ is a measure   & Yes & No (But countably additive)\\
        $ \mu(E)=0 \Rightarrow \int_E f \, d\mu=0$ &  Yes  &  Yes \\
        $f=0\;\, a.e.\Rightarrow \int f \, d\mu=0$ & Yes  & Yes   \\
        $\int f\, d\mu=0 \Rightarrow f=0 \;\, a.e.$ & Yes  & No \\
        \hline
     \end{tabular}
    \item Dominated convergence theorem (DCT)
    \begin{itemize}
        \item $\{f_n\}$ : sequence in $\LL^1$. let $f$ be a measurable function. If $f_n\rightarrow f \; a.e.$ and \\$|f_n|\leq g \;\, a.e.\forany n\in \N$ for some $g\in \LL^1$, then $f$ is integrable and $\int f \, d\mu= \lim_n \int f_n \, d\mu$ 
    \end{itemize}
    \item Generalized DCT : $\{f_n\}, \{g_n\}$ : sequences in $\LL^1$. let $f,g$ be integrable functions.\\If $f_n\rightarrow f \; a.e.\;,\; g_n\rightarrow g\; a.e.\;,\int g_n \, d\mu\rightarrow \int g\, d\mu$ and $\; |f_n|\leq g_n \forany n\in \N$ \\then $\int f \, d\mu= \lim_n \int f_n \, d\mu$ 
   
    \item[(Note)] Summary for classical results about interchanging limit and integral
    \\ (All Functions are at least measurable on the following statements) 
    \begin{itemize}
        
        \item MCT
        \begin{enumerate}
            \item $\{f_n\}$ nonnegative and increasing. $\Rightarrow \lim_n \int f_n \, d\mu=\int \lim_n f_n \, d\mu$
            \item $\{f_n\}$ nonnegative and $f_n\nearrow f$ a.e. $\Rightarrow \lim_n \int f_n \, d\mu=\int f \, d\mu$
            
        \end{enumerate}
        \item Fatou's lemma
        \begin{enumerate}
            \item $\{f_n\}$ nonnegative $\Rightarrow \int \liminf f_n \,d\mu\leq \liminf \int f_n \, d\mu$
            \item $\{f_n\}$ nonnegative and $f_n\rightarrow f\; a.e. \Rightarrow \int f \,d\mu\leq \liminf \int f_n \, d\mu$
            
        \end{enumerate}
        \item DCT
        \begin{enumerate}
            \item $\{f_n\}$ integrable, $f_n\rightarrow f \;\, a.e.$ and $|f_n|\leq g \;\, a.e.\forany n$ for some integrable $g$ \\$\Rightarrow f$ is integrable and $\int f \, d\mu= \lim_n \int f_n \, d\mu$
    
        \end{enumerate}
        \item Additional Results
        \begin{enumerate}
            \item $\{f_n\}$  integrable, $f_n\geq 0 \;\, a.e.\forany n$. and $f_n\nearrow f$ a.e. $\Rightarrow \lim_n \int f_n \, d\mu=\int f \, d\mu$
            \item $\{f_n\}$ increasing. $f_n\nearrow f$ a.e. and $f_n\geq g\forany n$ for some integrable $g\\ \Rightarrow \lim_n \int f_n \, d\mu=\int f \, d\mu$
            \item $\{f_n\}$  integrable and $f_n\geq 0 \;\, a.e.\forany n$. $\Rightarrow \int \liminf f_n \,d\mu\leq \liminf \int f_n \, d\mu$
            \item $\{f_n\}$  integrable, $f_n\geq 0 \;\, a.e.\forany n$. and $f_n\rightarrow f\; a.e. \Rightarrow \int f \,d\mu\leq \liminf \int f_n \, d\mu$
            \item $\{f_n\}$ integrable, $f_n\rightarrow f \;\, a.e.$ for some integrable $f$, and $\exist \{g_n\}$ integrable s.t. $|f_n|\leq g_n\forany n$ where $g_n\rightarrow g\;\, a.e.$ \& $\int g_n \, d\mu\rightarrow \int g\, d\mu$ for some integrable $g$ \\ $\Rightarrow \int f \, d\mu= \lim_n \int f_n \, d\mu$
            
        \end{enumerate}
    \end{itemize}
    \item Approximation in $\LL^1$
    \begin{itemize}
        \item $f\in \LL^1$. $\{\phi_n\}$ is the sequence of simple functions obtained by the Simple approximation theorem. ($\phi_n\rightarrow f$ and $|\phi_n|\leq |f|$).  Then we get \\ $\int f \, d\mu=\lim_n \int \phi_n \,d\mu$ \;and\; $\lim_n\int |f-\phi_n|\,d\mu=0$
    \end{itemize}
    \item $\{f_n\}$ : seq. in $\LL^1.$ and $f\in \LL^1$ s.t. $f_n\rightarrow f\;\, a.e.$ \\Then $\int |f_n-f|\,d\mu\rightarrow 0\Leftrightarrow \int |f_n|\, d\mu\rightarrow \int |f|\, d\mu$ 
\end{itemize}
\clearpage

\subsection{Concrete Examples}
\smallskip
\begin{itemize}
    \item The case of counting measures on $\N$
    \begin{itemize}
        \item Consider measure space $(\N, \mathcal{P}(\N)),\mu$ where $\mu$ is the counting measure. For any measurable function $f:\N \rightarrow \R$, $f$ can be regarded as a sequence $a_n=f(n)$
        \begin{enumerate}
            \item If $f\geq 0$ then $\int f \, d\mu=\sum_n a_n$
            \item If $f\in\LL^1$ then $\int f\, d\mu=\sum_n a_n$ and the series on RHS converges absolutely.
        \end{enumerate} 
    \end{itemize}
    \item The case of Dirac measures
    \begin{itemize}
        \item Consider measure space $(X, \mathcal{P}(X), \delta_x)$ for a fixed $x\in X$. \\For any $f:X\rightarrow \R$, we have $\int f\, d\delta_x=f(x)$
    \end{itemize}
    \item The case of the Lebesgue measure on $\R$
    \begin{itemize}
        \item[\rmk] Every Borel measurable function is Lebesgue measurable
        \item Every Riemann integrable function $f:[a,b]\rightarrow \R$ is Lebesgue intebrable and the result of integrals for both integrations are the same. i.e. $\int_{[a,b]}f \, dm=\int_a^b f(x)\, dx$
        \item (Characterization of Riemann integrability) For a bounded function $f: [a,b]\rightarrow \R$, \\ $f$ is Riemann integrable $\Leftrightarrow f$ is continuous $m-a.e.$ where $m$ is the Lebesgue measure
        \item (Generalize to improper integral) If $f:[0,\infty)\rightarrow \R$ satisfies that $f$ is Riemann integrable on $[0,b]\forany b>0$ and $lim_{b \to \infty}\int_0^b f(x)dx$ exists, i.e. the improper integral $\int_0^\infty f(x)dx$ exists, then $f$ is Lebesgue measurable and $\int_{[0,\infty)} f\, dm=\int_0^\infty f(x)dx$ provided $f$ is nonnegative or Lebesgue integrable. 
        \item (Interchanging partial differentiation and integration) Consider a bivariate function $f:[0,\infty)\times [a,b]\rightarrow \R$ s.t. $f(\cdot, y):[0,\infty)\rightarrow \R$ is integrable for each $y\in [a,b]$. Define $F(y)=\int_0^\infty f(x,y)dx$
        \begin{enumerate}
            \item Suppose $\exist g\in \LL^1$ s.t. $|f(x,y)|\leq g(x)\forany x,y$. \\Then $lim_{y\to y_0}f(x,y)=f(x,y_0) \forany x\Rightarrow lim_{y\to y_0}F(y)=F(y_0)$. \\ In particular, $f(x,\cdot)$ is continuous $\forany x \Rightarrow F$ is continuous.
            \item Suppose $\frac{\partial f}{\partial y}$ exists on $(a,b)$ and $\exist g\in \LL^1$ s.t. $|\frac{\partial f}{\partial y}(x,y)|\leq g(x)\forany x,y$ \\ Then $F$ is diff.able on $(a,b)$ and $F'(y)=\frac{\partial}{\partial y} \int_0^\infty f(x,y) dx= \int_0^\infty \frac{\partial}{\partial y}f(x,y) dx$  
        \end{enumerate}
  
    \end{itemize}
    \item The case of measure coming from function called density
    \begin{itemize}
        \item $(X, \A, \mu)$ : measure space. $\nu$ is the measure given by $\nu(E)=\int_E f\, d\mu\forany E\in \A$ for some $f\in \LL^+$ which is called as the density. \\ For any $g\in \LL^+$ or $g\in \LL^1(\nu)$, \, we have $\int g \, d\nu=\int fg \, d\mu$
    \end{itemize} 
    \item A remark on the completeness of measure
    \begin{itemize}
        \item Suppose$(X, \overline{\A}, \overline{\mu})$ is the completion of $(X,\A, \mu)$. If $f:X\rightarrow \R$ is $\overline{\A}$-measurable then $\exist g:X\rightarrow \R$ s.t. $g=f \;\, \overline{\mu}-a.e.$ and $g$ is $\A$-measurable.
    \end{itemize}
\end{itemize}
\clearpage

\section{Construction of Measures}
\smallskip
\begin{itemize}
    \item [*] Algebra $\A$ : a collection of sets containing the whole set $X$ and closed under taking complement or finite union \& intersection
    \item [*] Premeasure $\mu$ : a set function on an an algebra, which is nonnegative and countably additive $\mu(\phi)=0$. 
    \item[\rmk] Algebra is not closed under taking countable union, so countable additivity of premeasure $\mu$ is represented as $\mu(\bigcup_n A_n)=\sum_n \mu(A_n)$ for a disjoint $\{A_n\}\seq \A$ s.t. $\bigcup_n A_n \in \A$. 
    \item[\rmk] A premeasure $\mu$ is called $\sigma$-finite if $X=\bigcup_n X_n$ with $\{X_n\}\subset \A$ and $\mu(X_n)<\infty \forany n\in \N$
    \item [*] Outer measure $\mu^*$ 
    \begin{itemize}
        \item For a premeasure $\mu$ on an algebra $\A\subset \mathcal{P}(X)$, the outer measure $\mu^* : \mathcal{P}(X)\rightarrow [0,\infty]$ is defined by $\mu^*(E)=\inf \{\sum_n \mu(A_n) : A_n\in \A,\; \bigcup_n A_n$ covers $E\}$ 
    \end{itemize}
    \item Elementary properties of outer measure
    \begin{enumerate}
        \item $\mu^*(\phi)=0$
        \item (Monotonicity) For any $A, B\subset X$, if $A\subset B$ then $\mu^*(A)\leq \mu^*(B)$
        \item (Countable Subadditivity) For any $\{A_n\}\subset \mathcal{P}(X)$, $\mu^*(\bigcup_n A_n)\leq \sum_n \mu^*(A_n)$
    \end{enumerate}
    \item [*] Caratheodory condition
    \begin{itemize}
        \item $\mu^*$ : the outer measure associated to a premeasure $\mu$ on an algebra $\A\subset \mathcal{P}(X)$. $E\subset X$ is said to be $\mu^*$-measurable if $E$ satisfies the `Catheodory condition' below : \\
        $\mu^*(A)=\mu^*(A\intersect E)+\mu^*(A\intersect E^C)$ for any $A\subset X$
        \item The collection of all $\mu^*$-measurable sets is denoted by $\mathcal{M}^*$
    \end{itemize} 
    \item Caratheodory Extension Theorem
    \begin{itemize}
        \item $\mu^*$ : the outer measure associated to a premeasure $\mu$ on an algebra $\A\subset \mathcal{P}(X)$. \\ The followings are true. \quad (a) $\mathcal{M}^*$ is a $\sigma$-algebra. \; (b) $\mu^*|_{\mathcal{M}^*}$ is a measure. \\ (c) $\mu^*|_\A = \mu$ \; i.e. \, $\mu^*$ is indeed an extension of $\mu$ \quad (d) $\A\subset \mathcal{M}^*$ 
        \item In particular, if we denote $\mathcal{M}=\sigma(\A)$, then $\mathcal{M}\subset \mathcal{M}^*$ by the result of (d). \\ Define a measure $\tilde \mu$ on $\mathcal{M}$ by $\tilde\mu=\mu^*|_\mathcal{M}$. If $\mu$ is a $\sigma$-finite premeasure \\then $\tilde\mu$ is the unique extension of $\mu$ which is a measure on $\mathcal{M}$
    \end{itemize}
\end{itemize}
\clearpage

\subsection{The Lebesgue Measure on $\R$}
\smallskip
\begin{itemize}
    \item Building an appropriate algebra to construct Lebesgue measure on $\R$
    \begin{itemize}
        \item $\mathcal{I}=\{(a,b]:-\infty<a<b<\infty\}$,\; $\mathcal{J}=\{(-\infty, b]:b\in \R\}$, \; $\mathcal{K}=\{(a,\infty):a\in \R\}$ \\
        Every element of $\mathcal{I}\union \mathcal{J}\union \mathcal{K}$ is said to be h-interval. \;(`h' stands for `half-open')\\ $\A$ is defined as the collection of finite unions of disjoint h-intervals.
        \item $\A$ above is an algebra.
        \item $\A$ generates a Borel $\sigma$-algebra \; i.e. \, $\sigma(\A)=\Borel$
    \end{itemize}
    \item Constructing a premeasure which extends a length function.
    \begin{itemize}
        \item Define $\mu : \A \rightarrow [0,\infty]$ by $\mu(\bigcup_{k=1}^n I_k)=\sum_{k=1}^n length(I_k)$ where $I_k$'s are disjoint h-intervals. Note that length of every ray is $\infty$. 
        \item $\mu$ is well-defined premeasure on $\A$
    \end{itemize} 
    \item Lebesgue measure on $\R$
    \begin{itemize}
        \item The Lebesgue measure $m$ on $(\R,\Borel)$ is the unique extension of $\mu$ on $\Borel$, which is  guaranteed by the Caratheodory Extension Thm. 
    \end{itemize}
    \item Elementary properties of Lebesgue measure
    \begin{itemize}
        \item $m(I)=length(I)$ for any interval $I\subset \R$ (For any ray, length is measured as $\infty$) 
        \item $m(E+x)=m(E) \;\forany E\in \Borel$ and $\forany x\in \R$ \; i.e.\, $m$ is translation invariant.
    \end{itemize}
\end{itemize}
\smallskip

\subsection{The $\sigma$-algebra of Lebesgue Measurable sets $\LL(\R)$ and $\mathcal{M}^*$}
\smallskip
\begin{itemize}
    \item Assume same condition with the Caratheodory Extension Thm  ; \\ $\mu^*$ : the outer measure associated to a premeasure $\mu$ on an algebra $\A\subset \mathcal{P}(X)$. \\ Then $(X, \mathcal{M}^*, \mu^*|_{\mathcal{M}^*})$ is a complete measure space.
    \item Assume that we're in the situation of constructing Lebesgue measure with premeasure $\mu$. Take $\varepsilon>0$. For any $E\in \mathcal{M}^*$, $\exist F\closed \R, \U\open \R$ s.t. $F\subset E\subset \U$ and $\mu^*(\U\diff F)<\varepsilon$. Moreover, $\exist$ a $F_\sigma$ set $F$ and a $G_\delta$ set $G$ s.t. $F\subset E\subset G$ and $\mu^*(G\diff F)=0$. 
    \item The $\sigma$-algebra $\mathcal{M}^*$ appearing in the construction of Lebesgue measure is the same as the $\sigma$-algebra of Lebesgue measurable sets $\LL(\R)$ defined by a completion of $\Borel$. The outer measure $\mu^*$ restricted to $\mathcal{M}^*$ in the construction of Lebesgue measure is indeed the extended Lebesgue measure $\overline{m}$ on $\LL(\R)$.
    \item [\sq] Regularity of Lebesgue measure $m$
    \begin{itemize}
        \item Let $\varepsilon>0$. For $E\in \Borel$, $\exist F\closed \R, \U\open \R$ s.t. $F\subset E\subset \U$ and $m(\U\diff F)<\varepsilon$. \\ Moreover, $\exist F, G\in \Borel$ s.t. $F\subset E\subset G$ and $m(G\diff F)=0$.
    \end{itemize} 
\end{itemize}
\smallskip

\subsection{Probability Borel Measure and Distribution Function}
\smallskip
\begin{itemize}
    \item Construction of Probability Borel Measure $\mu_F$ from a distribution function $F$
    \begin{itemize}
        \item Let $F : \R\rightarrow \R$ be a monotone increasing right-continuous function with \\$F(-\infty)=\lim_{x \to -\infty} F(x)=0$ and $F(\infty)=\lim_{x to \infty} F(x)=1$. Also consider algebra $\A$ used in the construction of Lebesgue measure.
        \item Define $\mu : \A \rightarrow [0,1]$ by $\mu(\phi)=0$ and $\mu(\bigcup_{k=1}^n (a_k, b_k])=\sum_{k=1}^n F(b_k)-F(a_k)$ \\ Here, if $b_k=\infty$ then regard $(a_k, b_k]$ as $(a_k, \infty)$.
        \item $\mu$ above is a premeasure on $\A$. Since $\mu$ is a finite premeasure and $\sigma(\A)=\Borel$, thanks to the Caratheodory Extension Thm, there is a unique extension of $\mu$ which is a measure on $\Borel$.
        \item We denote such measure as $\mu_F$, which is a probability Borel measure.\\ $\mu_F(a,b]=F(b)-F(a) \;\forany -\infty<a<b<\infty$ and $\mu_F(-\infty, x]=F(x) \;\forany x\in \R$.
        \item Using same logic, we can construct a unique Borel Measure from a distribution-like function $F$ s.t. $F(\infty)=\lim_{x to \infty} F(x)\leq 1$.
    \end{itemize}
\end{itemize}
\bigskip

\section{Product Measures and The Fubini-Tonelli Theorem}
\smallskip
\subsection{Construction of Product Measure}
\smallskip
\begin{itemize}
    \item [*] Product $\sigma$-algebra
    \begin{itemize}
        \item $(X,\A),(Y,\B)$ : measurable spaces. The product $\sigma$-algebra $\A\otimes \B$ is defined by the $\sigma$-algebra on $X\times Y$ generated by $\A\times \B=\{A\times B: A\in \A, \;B\in \B\}$ \\i.e.\; $\A\otimes \B=\sigma(\A\times\B)$ 
        \item Every set of the form $A\times B: A\in \A, \;B\in \B$ is called as (measurable) rectangles. 
    \end{itemize}
    \item If $X$ and $Y$ are topological spaces satisfying second countability axiom \; i.e.\, each $X$ and $Y$ has a countable basis, then $\B(X)\times\B(Y)=\B(X\times Y)$ \quad (Note that for a topological space $X$, $\B(X)$ is defined as a $\sigma$-algebra generated by a collection of all open sets in $X$)
    \item [\sq] $\Borel\otimes \Borel = \B(\R^2)$ and $\mathcal{P}(\N)\otimes\mathcal{P}(\N)=\mathcal{P}(\N^2)$
    \item Product measure
    \begin{itemize}
        \item $(X,\A, \mu),(Y,\B,\nu)$ : measure spaces. A product measure $\mu\times \nu$ is a measure on $\A\otimes \B$ satisfying $\mu\times\nu(A\times B)=\mu(A)\nu(B)\forany A\in \A, B\in \B$ (existence of such measure is guaranteed by the Caratheodory Extension Thm.)
        \item In addition, if $\mu$ and $\nu$ are both $\sigma$-finite, then $\mu\times \nu$ is uniquely determined.
    \end{itemize} 
\clearpage
    \item Examples of product measure
    \begin{enumerate}
        \item The Lebesgue measure on $\R^2$
        \begin{itemize}
            \item The product measure of Lebesgue measure $m^2=m\times m$ is an extension of area function on $\B(\R^2)$ in the sense that for any real interval $I$ and $J$, we get\\ $m^2(I\times J)=m(I)m(J)=length(I)\times length(J)=area(I\times J)$  
        \end{itemize}
        \item The Counting measure on $\N^2$
        \begin{itemize}
            \item The product measure $\mu^2=\mu\times \mu$ where $\mu$ is a counting measure on $\mathcal{P}(\N)$ is also a counting measure on $\N^2$
        \end{itemize}
        \item The product of probability Borel measures
        \begin{itemize}
            \item If each $\mu$ and $\nu$ is a probability Borel measures on $(\R, \Borel)$ then there is a uniquely determined probability Borel measure $\mu\times \nu$ on $(\R^2, \B(\R^2))$
            \item If $X$ and $Y$ are independent random variables with distribution $X\sim \mu$ and \\$Y\sim \nu$ respectively, \; i.e. $P(X\in B)=\mu(B)$ and $P(Y\in B)=\nu(B)\;\forany B\in \Borel$ \\ then a random vector $(X,Y)$ has a distribution $(X,Y)\sim \mu\times \nu$ so that \\
            $P((X,Y)\in A\times B)=P(X\in A)P(Y\in B)\;\forany A,B\in \Borel$
        \end{itemize} 
    \end{enumerate}
\end{itemize}
\smallskip
\subsection{Fubini-Tonelli Theorem}
\smallskip
\begin{itemize}
    \item [*] The concept of sections
    \begin{enumerate}
        \item $X,Y$ : sets. For any $E\subset X\times Y$ and $x\in X, y\in Y$, $x$-section and $y$-section of $E$ are defined by $E_x=\{y\in Y : (x,y)\in E\}$, $E^y=\{x\in X : (x,y)\in E\}$
        \item $f : X\times Y\rightarrow \R$. $x$-section and $y$-section of $f$ are defined by $f_x : Y\rightarrow \R$, $f^y:X\rightarrow \R$ and $f_x(y)=f(x,y)\forany y\in Y$, $f^y(x)=f(x,y)\forany x\in X$
        \item [\rmk] $(\chi_E)_x=\chi_{E_x}$ (as a map from $Y$ to $\R$) / $(\chi_E)^y=\chi_{E^y}$ (as a map from $X$ to $\R$)
    \end{enumerate}
    \item Measurability of sections
    \begin{itemize}
        \item $(X, \A), (Y,\B)$ : measurable spaces
        \begin{enumerate}
            \item If $E\in \A\otimes \B$ then $E_x\in \B$ and $E^y\in \A \;\forany x\in X, y\in Y$
            \item If $f:X\times Y\rightarrow \R$ is $\A\otimes \B$-measurable then $f_x$ is $\B$-measurable and $f^y$ is $\A$-measurable \;$\forany x\in X, y\in Y$
        \end{enumerate}
    \end{itemize}
    \item [*] Monotone Class
    \begin{itemize}
        \item $X$ : a set. $\A\subset \mathcal{P}(X)$ is said to be a monotone class if \\$E_n\in \A, E_n\subset E_{n+1}\forany n\Rightarrow \bigcup_n E_n\in \A$ / $E_n\in \A, E_n\supset E_{n+1}\forany n\Rightarrow \bigcap_n E_n\in \A$
        \item [\rmk] Every $\sigma$-algebra is a monotone class. An intersection of monotone classes on the same set is a monotone class. 
    \end{itemize}
    \item The monotone class lemma
    \begin{itemize}
        \item If $\A\subset \mathcal{P}(X)$ is an algebra then $\C(\A)=\sigma(\A)$ where $\C(\A)$ denotes the smallest monotone class containing $\A$.
    \end{itemize}
\clearpage
    \item $(X, \A, \mu), (Y,\B, \nu)$ : $\sigma$-finite measure spaces. For $E\in \A\otimes \B$, the followings are satisfied.
    \begin{enumerate}
        \item $x\mapsto \nu(E_x)$ is a $\A$-measurable function / $y\mapsto \mu(E^y)$ is a $\B$-measurable function
        \item $\mu\times\nu(E)=\int \nu(E_x)\,d\mu(x)=\int \mu(E^y)\,d\nu(y)$
    \end{enumerate}
    \item Fubini-Tonelli Theorem
    \begin{itemize}
        \item $(X, \A, \mu), (Y,\B, \nu)$ : $\sigma$-finite measure spaces. $f: X\times Y\rightarrow \R$
        \begin{enumerate}
            \item (Tonelli) \\ If $f$ is nonnegative $A\otimes B$-measurable function, then $g(x)=\int f_x\,d\nu$ is nonnegative $\A$-measurable, $h(y)=\int f^y\,d\mu$ is nonnegative $\B$-measurable, and the equation ($*$) holds true.
            \item (Fubini) \\ If $f$ is integrable function then $f_x$ is integrable for almost all $x\in X$, $f^y$ is integrable for almost all $y\in Y$, and $g(x)=\int f_x\,d\nu$ \& $h(y)=\int f^y\,d\mu$ are integrable. Moreover the equation ($*$) holds true. $$(*)\; \int_{X\times Y} f\, d(\mu\times \nu)=\int_{X}\int_{Y} f(x,y)\,d\nu(y)d\mu(x)=\int_{Y}\int_{X} f(x,y)\,d\mu(x)d\nu(y)$$
            \item[\rmk] The meaning of statement of Fubini Thm is : \\ If $f\in \LL^1(\mu\times\nu)$ \; i.e.\, $\int |f|\, d(\mu\times \nu)<\infty$ then $f_x\in \LL^1(\nu)$ \; i.e.\, $\int |f_x|\,d\nu<\infty$ for $\mu-a.e. \; x$ \, and\, $f^y\in \LL^1(\mu)$ \; i.e.\, $\int |f^y|d\mu<\infty$ for $\nu-a.e. \; y$
        \end{enumerate}
    \end{itemize}
    \item [\sq] Fubini-Tonelli Thm for Probability Theory
    \begin{itemize}
        \item $X, Y$ : independent random variables with distribution $X\sim \mu$ and $Y\sim \nu$ \\ If a Borel measurable function $f:\R^2\rightarrow \R$ satisfies $f\geq 0$ or $E|f(X,Y)|<\infty$ then \\ $$E[f(X,Y)]=\int_Y\int_X f(x,y)\, d\mu(x)d\nu(y)=\int_X\int_Y f(x,y)\, d\nu(y)d\mu(x)$$
    \end{itemize}
    \item Useful result about Lebesgue integral
    \begin{itemize}
        \item If Borel measurable $f:\R\rightarrow \R$ is nonnegative or Lebesgue integrable function, then $\int f(x+\alpha)\, dm(x)=\int f(x)\, dm(x)$ and $\int f(\alpha x)\, dm(x)=\alpha^{-1}\int f(x)\, dm(x) \;\forany \alpha\neq 0$ 
    \end{itemize}
    \item If $T : \R^2\rightarrow \R^2$ is an invertible linear map and $f :\R^2\rightarrow \R$ is Borel measurable function, then we have $\int f\, dm^2=|det(T)|\int f\circ T\, dm^2$
    \item Properties of the Lebesgue measure $m^2$ on $\R^2$
    \begin{itemize}
        \item Rotation invariance ; $m^2(R(E))=m^2(E)\;\forany E\in \B(\R^2)$ where $R$ is a rotation map
        \item Translation invaraince ; $m^2(E+x)=m^2(E)\; \forany E\in \B(\R^2), x\in \R^2$ 
    \end{itemize}
\end{itemize}
\clearpage

\section{The Spaces $L^1$ and $L^2$}
\smallskip
\begin{itemize}
    \item [*] Banach Space
    \begin{itemize}
        \item A complete normed vector space is said to be a Banach space
    \end{itemize}
    \item [*] $L^1$-norm
    \begin{itemize}
        \item $(X, \A, \mu)$ : a measure space. $\|\cdot\|_1$ on $\LL^1=\LL^1(X, \A, \mu)$ is defined as $\|f\|_1=\int |f|\, d\mu$
        \item In order to satisfy the defining properties of norm, introduce equivalence relation on $\LL^1$ given as $f\sim g \Leftrightarrow f=g \; a.e.$ 
    \end{itemize}
    \item [*] The space $L^1$
    \begin{itemize}
        \item $(X, \A, \mu)$ : a measure space. Define $L^1=L^1(X,\A, \mu)$ by $L^1=\LL^1/\sim$ i.e. $L^1$ is the space of equivalence classes in $\LL^1$ w.r.t. $\sim$ above. 
        \item[\rmk] $L^1$ is a normed space with $\|\cdot \|_1$. $L^1$ identifies $f,g\in \LL^1$ whenever $f=g\; a.e.$
    \end{itemize}
    \item Riesz-Fisher Theorem
    \begin{itemize}
        \item $L^1$ is a Banach space.
    \end{itemize}
    \item The effect of completion of measure spaces for $L^1$
    \begin{itemize}
        \item $(X, \overline{A}, \overline{\mu})$ is the completion of $(X, \A, \mu)$. Then we have $L^1(X, \overline{\mu})=L^1(X, \mu)$ as Banach spaces. i.e. there is a norm preserving linear bijection between two spaces.
        \item[\rmk] We cannot distinguish $L^1(\R, \LL(\R), m)$ from $L^1(\R, \Borel, m)$, which is why we do not need to consider the $\sigma$-algebra of Lebesgue measurable sets in reality.
    \end{itemize}
    \item [*] Hilbert space
    \begin{itemize}
        \item An inner product space being complete as a normed space is called as a Hilbert space.
    \end{itemize}
    \item The space $L^2$
    \begin{itemize}
        \item $(X,\A, \mu)$ : a measure space. Define $\LL^2=\big\{f:X\rightarrow \R\union \{\pm \infty\}\;| \int |f|^2\, d\mu<\infty \big\}$
        \item[\rmk] $\LL^2$ is a vector space.
        \item $L^2$ is defined as $L^2=\LL^2/\sim$ where $\sim$ is the equivalence relation of being equal almost everywhere.
        \item Inner product on $L^2$ and the induced $L^2$-norm is given as  $\langle f,g \rangle = \int fg \, d\mu $ and $\|f\|_2 = \big( \int |f|^2 \, d\mu \big)^{1/2}$ 
    \end{itemize}
    \item Approximation in $L^1$ and $L^2$
    \begin{enumerate}
        \item For any $f\in L^1$ and $\varepsilon>0$, $\exist$ a simple function $\phi\in L^1$ with $\|f-\phi\|_1<\varepsilon$
        \item For any $f\in L^2$ and $\varepsilon>0$, $\exist$ a simple function $\phi\in L^2$ with $\|f-\phi\|_2<\varepsilon$
    \end{enumerate}
\end{itemize}
\subsection{Concrete Cases}
\smallskip
\begin{itemize}
    \item The case of $([0,1],\mathcal{B}([0,1]), m)$, which is a probability space.
    \begin{itemize}
        \item $L^2([0,1], m)\subset L^1([0,1],m)$ with $\|f\|_1\leq \|f\|_2 \;\forany f\in L^2([0,1],m)$
    \end{itemize}
    \item The case of $(\N, \mathcal{P}(\N), \mu)$ where $\mu$ is a counting measure.
    \begin{itemize}
        \item Denote $L^p(\N, \mu)$ by $\ell^p$ and call it the little $L^p$ space or the sequential $L^p$ space.
        \item $\ell^1 \subset \ell^2$ with $\|\{a_n\}\|_2\leq \|\{a_n\}\|_1 \; \forany \{a_n\}\in \ell^1$
    \end{itemize}
    \item The case of $(\R, \Borel, m)$
    \begin{itemize}
        \item There is no inclusion between the spaces $L^1(\R, \Borel, m)$ and $L^2(\R, \Borel, m)$
    \end{itemize}
\end{itemize}
\bigskip

\section{Basic Fourier Analysis}
\smallskip
\subsection{Integration of complex-valued functions}
\smallskip
\begin{itemize}
    \item Componentwise measurability implies measurability of vector-valued function
    \begin{itemize}
        \item $(X, \A)$ : a measurable space. $f:X\rightarrow \R^2$. Componenetwise representation of $f$ is $(f_1, f_2)$. Then $f_1, f_2$ are $\A$-measurable $\Leftrightarrow f$ is measurable. \\ i.e. \;$f_1^{-1}(B_1),\, f_2^{-1}(B_2)\in \A \;\forany B_1, B_2\in \Borel \Leftrightarrow f^{-1}(B)\in \A \;\forany B\in \mathcal{B}(\R^2)$
    \end{itemize}
    \item[*] Measurability of complex-valued function
    \begin{itemize}
        \item $(X, \A, \mu)$ : a measure space. $f:X\rightarrow \mathbb{C}$
        \begin{enumerate}
            \item $f$ is said to be $\A$-measurable if $Re(f)$ and $Im(f)$ are $\A$-measurable.
            \item $f$ is said to be integrable if $Re(f)$ and $Im(f)$ are integrable. \\In this case we define integral of $f$ by $\int f\, d\mu := \int Re(f)\, d\mu + i\int Im(f)\,d\mu$
            \item[\rmk]  If $f$ is a measurable complex-valued function,\\ then $f$ is integrable $\Leftrightarrow |f|$ is integrable.
        \end{enumerate}
    \end{itemize}
    \item Elementary properties of integral of complex-valued functions
    \begin{itemize}
        \item Let $f,g : X\rightarrow \mathbb{C}$ be integrable functions
        \begin{enumerate}
            \item For $\alpha, \beta \in \mathbb{C}$, we have $\int \alpha f +\beta g \, d\mu = \alpha \int f \, d\mu + \beta \int g\, d\mu$
            \item $|\int f\, d\mu|\leq \int |f| \, d\mu$
        \end{enumerate}
    \end{itemize}
    \item[*] $L^1$ and $L^2$ spaces of complex-valued functions.
    \begin{itemize}
        \item $(X, \A, \mu)$ : a measure space. we define $\LL_{\mathbb{C}}^1$ and $\LL_{\mathbb{C}}^2$ spaces as below \begin{equation*}
            \LL_{\mathbb{C}}^1(X, \A, \mu)=\big\{f:X\rightarrow \mathbb{C}\, |\, \int |f| \, d\mu<\infty  \big\}, \quad \LL_{\mathbb{C}}^2(X, \A,\mu) =\big\{f:X\rightarrow \mathbb{C}\, |\, \int |f|^2 \, d\mu<\infty  \big\}
        \end{equation*}
        \item We also define $L_{\mathbb{C}}^1$ and $L_{\mathbb{C}}^2$ by $L_{\mathbb{C}}^1:=\LL_{\mathbb{C}}^1 /\sim$ and $L_{\mathbb{C}}^2:=\LL_{\mathbb{C}}^2 /\sim$
        \item $L^1$ and $L^2$ spaces are complex normed spaces with the norms \begin{equation*}
            \|f\|_1=\int |f| \ d\mu, \quad \|f\|_2=\Big(\int |f|^2\, d\mu\Big)^2
        \end{equation*}
        \item Note that $L^2$ is also a complex inner product space with inner product \begin{equation*} \langle f, g \rangle = \int f\overline{g}\, d\mu \end{equation*}
    \end{itemize}
    \item Orthonormal family in $L^2$
    \begin{itemize}
        \item \begin{equation*}
            \frac{1}{2\pi} \int_{-\pi}^\pi e^{inx}e^{-imx}\, dx = \begin{cases}
                1 & (n=m) \\ 0 & (n\neq m)
            \end{cases}
        \end{equation*}
        \item $\{e^{inx} : n\in \mathbb{Z}\}$ is an orthonormal family in $L^2([-\pi, \pi], \frac{1}{2\pi}m)$ where $\frac{1}{2\pi}m$ is the normalized Lebesgue measure on $[-\pi, \pi]$
    \end{itemize}
\end{itemize}
\smallskip
\subsection{Fourier series of periodic funcitons}
\smallskip
\begin{itemize}
    \item[*] Fourier coefficient and Fourier series
    \begin{itemize}
        \item Let $f:[-\pi, \pi]\rightarrow \mathbb{C}$ be an integrable funtion w.r.t. the Lebesgue measure. 
        \begin{enumerate}
            \item For each $n\in \mathbb{Z}$, the $n$-th Fourier coefficient of $f$ is defined by \begin{equation*}
                \hat{f}(n)=\frac{1}{2\pi}\int_{-\pi}^\pi f(x)e^{-inx}\, dx=\langle f, e_n \rangle
            \end{equation*}
            \item Fourier series of $f$ is defined as the following formal series \begin{equation*}
                \sum_{n=-\infty}^\infty \hat{f}(n)e^{inx}
            \end{equation*}
            \item Partial sums of the Fourier series of $f$ is denoted by $S_N(f)$, which is given as \begin{equation*}
                S_N(f)(x)=\sum_{n=-N}^N \hat{f}(n)e^{inx}
            \end{equation*}
        \end{enumerate}
        \item[\rmk] Given the integrability of $f$, $\hat{f}(n)$ is well-defined for every $n\in \mathbb{Z}$
        \item[\rmk] Fourier series is called 'formal' since we don't know its convergence at this moment.  
    \end{itemize}
    \item[*] Trigonometric series and polynomial
    \begin{itemize}
        \item Trigonometric series is the series of the following form ; $\sum_{n=-\infty}^\infty c_n e^{inx}$
        \item Trigonometric polynomial is a special case of trigonometric series with $c_n=0$ for $|n|>N$ , i.e. $\sum_{n=-N}^N c_n e^{inx}$
        \item If $c_N\neq 0$ or $c_{-N}\neq0$ then $N$ is called the degree of the trigonometric polynomial
    \end{itemize}
    \item[\rmk] The main problem of this section is ``In what sense $S_N(f)$ converges to $f$ as $N\rightarrow \infty$''
\end{itemize}
\smallskip
\subsection{Convolutions and good kernels}
\smallskip
\begin{itemize}
    \item[*] Convolution
    \begin{itemize}
        \item For $2\pi$-periodic square-integrable functions $f$ and $g$ on $\R$, the convolution $f*g$ on $[-\pi, \pi]$ is defined as \begin{equation*}
            f*g (x)=\frac{1}{2\pi}\int_{-\pi}^\pi f(y)g(x-y)\,dy
        \end{equation*}
        \item[\rmk] By Cauchy-Schwarz inequality, convolution is well-defined.
        \item[\rmk] Loosely speaking, convolution is a kind of weighted average. 
    \end{itemize}
    \item Properties of convolutions
    \begin{itemize}
        \item For $2\pi$-periodic $f, g, h\in L^2$ and $\alpha, \beta \in \mathbb{C}$  \begin{enumerate}
            \item $f*g=g*f$
            \item $(\alpha f+\beta g)* h= \alpha f*h+\beta g*h$
            \item $\hat{f*g}=\hat{f}\cdot \hat{g}$ \,i.e.\, $\hat{f*g}(n)=\hat{f}(n)\hat{g}(n)\quad \forany n\in \mathbb{Z}$
        \end{enumerate}
    \end{itemize}
   
    \item[*] Dirichlet kernel 
    \begin{itemize}
        \item $D_N(x)=\sum_{n=-N}^N e^{inx}$ is called as the $N$-th Dirichlet kernel. 
        \item Simplified form is $D_N(x)=\frac{sin((N+1/2)x)}{sin(x/2)}$
    \end{itemize}
    \item For $2\pi$-periodic $f\in L^2$, we have $S_N(f)=f*D_N$
    \item[*] A family of good kernels
    \begin{itemize}
        \item A family of good kernels is a family $\{K_n\}_{n\in \N}$ of functions on $[-\pi, \pi]$ satisfying \begin{enumerate}
            \item (Normalized) $\frac{1}{2\pi}\int_{-\pi}^\pi K_n(x)\,dx=1 \quad \forany n\in \N$
            \item (Boundedness) $\{\|K_n\|_1\}_{n\in \N}$ is a bounded sequence of positive numbers.
            \item (Concentration at zero) $\forany \delta>0$, we have $\int_{\delta\leq |x|\leq \pi} |K_n(x)|\, dx\rightarrow 0$ as $n\rightarrow \infty$
        \end{enumerate}
    \end{itemize}
    \item Good kernels and convergence of convolution
    \begin{itemize}
        \item If $\{K_n\}_{n\in \N}$ is a family of good kernels and $f$ is continuous function on $[-\pi, \pi]$ then $f*K_n$ converges to $f$ uniformly on $[-\pi, \pi]$
    \end{itemize}
    \item Dirichlet kernels $\{D_n\}_{n\in\N}$ is not a family of good kernels. Hence it is difficult for us to hope that the partial sums $S_N(f)$ converges to $f$ uniformly.
    \item[*] Cesaro Summable 
    \begin{itemize}
        \item A sequence of complex numbers $\{c_n\}_n$ is said to be Cesaro summable to $c\in \mathbb{C}$ if the arithmetic mean of their partial sums converges to $c$
        \item For the case of Fourier series of $f$, ``the Fourier series of $f$ is Cesaro summalbe to $f$'' means that $\sigma_N(f)=\frac{S_0(f)+S_1(f)+\cdots+S_{N-1}(f)}{N}$ converges to $f$ as $N\rightarrow \infty$  
    \end{itemize}
    \item[*] Fejer kernel 
    \begin{itemize}
        \item $F_N=\frac{D_0+D+1+\cdots+D_{N-1}}{N}$ is called as the $N$-th Fejer kernel.
        \item Simplified form is $F_N(x)=\frac{\sin^2(Nx/2)}{N\sin^2(x/2)}$
    \end{itemize} 
    \item Fejer kernels $\{F_n\}_{n\in \N}$ is a fmily of good kernels.
    \item If $f$ is a continuous function on $[-\pi, \pi]$ with $f(-\pi)=f(\pi)$ then $f*F_n$ converges to $f$ uniformly on $[-\pi, \pi]$ i.e. the Fourier series of $f$ is uniformly Cesaro summable to $f$.
    \item[\sq] If $f$ is a continuous function on $[-\pi, \pi]$ with $f(-\pi)=f(\pi)$ then $f$ can be uniformly approximated by trigonometric polynomials.\; i.e. $\forany \varepsilon>0$, there is a trigonometric polynomial $p=f*F_N$ with large enough $N$ s.t. $|f(x)-p(x)|<\varepsilon \quad \forany -\pi\leq x\leq \pi$
\end{itemize}
\smallskip
\subsection{Convergence of Fourier series in $L^2$ space and Plancherel Thm}
\smallskip
\begin{itemize}
    \item Here we consider $L^2([-\pi, \pi])$ having $L^2$-norm defined as $\|f\|_2^2=\frac{1}{2\pi}\int_{-\pi}^\pi |f(x)|^2\, dx$. Note that for $f\in L^2([-\pi, \pi])$, we can always assume $f(-\pi)=f(\pi)$ since we identify functions which coincide almost everywhere.
    \item Fourier partial sum as the Best approximtaion
    \begin{itemize}
        \item If $f\in L^2([-\pi, \pi])$ then we have $\langle f-S_N(f), e_n \rangle=0 \; \forany |n|\leq N$ and \begin{equation*}  \quad 
            \|f-S_N(f)\|_2\leq \bigg\|f-\sum_{|n|\leq N}c_ne_n\bigg\|_2 \quad \forany c_n\in \mathbb{C}
        \end{equation*}
        \item[\rmk] It tells us that Fourier partial sum is the best approximation for a function in $L^2([-\pi, \pi])$ space among all trigonometric polynomials with same order in the sense of $L^2$-distance.
    \end{itemize} 
    \item For any $f\in L^2([-\pi, \pi])$ and $\varepsilon>0$, $\exists$ a $2\pi$-periodic continuous function $g$ on $[-\pi, \pi]$ s.t. $\|f-g\|_2<\varepsilon$
    \item Convergence of Fourier series in $L^2$
    \begin{itemize}
        \item For any $f\in L^2([-\pi, \pi])$ we have $\|f-S_N(f)\|_2\rightarrow 0$ as $N\rightarrow \infty$
    \end{itemize}
    \item Parseval's identity
    \begin{itemize}
        \item For any $f\in L^2([-\pi, \pi])$ we have the identity below \begin{equation*}
            \|f\|_2^2=\frac{1}{2\pi}\int_{-\pi}^\pi |f(x)|^2\, dx=\sum_{n\in \mathbb{Z}}|\hat{f}(n)|^2
        \end{equation*}
    \end{itemize}
    \item[(Ex)] Using Parseval's identity with $f(x)=x$ on $[-\pi, \pi]$, we can show $\sum_{n=1}^\infty \frac{1}{n^2}=\frac{\pi^2}{6}$
    \item[\sq]Riemann-Lebesgue Lemma (Easy version)
    \begin{itemize}
        \item For any $f\in L^2([-\pi, \pi])$ we have $\hat{f}(n)\rightarrow 0$ as $|n|\rightarrow \infty$
    \end{itemize} 
    \item Plancherel Theorem
    \begin{itemize}
        \item $\Phi : L^2([-\pi, \pi])\rightarrow \ell^2(\mathbb{Z})$ given by $f\mapsto \{\hat{f}(n)\}_{n\in \mathbb{Z}}$ is a linear isometric bijection
    \end{itemize} 
\end{itemize}
\bigskip

\section{The space $L^p$}
\smallskip
\begin{itemize}
    \item[*] $L^p$ space for $p\geq 1$
    \begin{itemize}
        \item $\LL^p:=\big\{f:X\rightarrow \R\union \{\pm \infty\}\,|\, \int |f|^p\, d\mu <\infty\big\}$
        \item[\rmk] $\LL^p$ is a vector space since $|f+g|^p\leq 2^{p-1}(|f|^p+|g|^p)\;(\because$ convexity of $x\mapsto |x|^p$ given $p\geq 1)$ and $|\alpha f|^p=|\alpha|^p|f|^p\;\forany \alpha\in \R$
        \item $L^p:=\LL^p/\sim$. \quad $L^p$-norm is $\|f\|_p=\Big(\int |f|^p \, d\mu\Big)^{\frac{1}{p}}$
        \item[\rmk] To show $L^p$-norm is indeed a norm, we need to show the triangle inequality.  
    \end{itemize}
    \item[*] Conjugate exponents
    \begin{itemize}
        \item For $p,q\geq1$, $p$ and $q$ are called as conjugate exponents if $\frac{1}{p}+\frac{1}{q}=1$
    \end{itemize}  
    \item Young's inequality
    \begin{itemize}
        \item $a,b>0$ and $1<p,q<\infty$. If $p$ and $q$ are conjugate exponents then $ab\leq \frac{a^p}{p}+\frac{b^q}{q}$
        \item Equality holds iff $a^p=b^q$ 
    \end{itemize}
    \item H\"older's inequality
    \begin{itemize}
        \item If $1<p,q<\infty$ are conjugate exponents and $f\in L^p\, ,\, g\in L^q$ then $fg\in L^1$ and \begin{equation*}
            \|fg\|_1=\int |fg|\, d\mu \leq \|f\|_p\|g\|_q
        \end{equation*}
        \item Equality holds iff $\alpha |f|^p=\beta |g|^q \; a.e.$ for some $\alpha, \beta\in \R$ s.t. $(\alpha, \beta)\neq (0,0)$  \\ If $\|f\|_p>0$ and $\|g\|_q>0$ then $\alpha=\|g\|_q^q$ and $\beta=\|f\|_p^p$
    \end{itemize}
    \item Minkowski's inequality
    \begin{itemize}
        \item For $1<p<\infty$, if $f,g\in L^p$ then $\|f+g\|_p\leq \|f\|_p+\|g\|_p$
        \item[\rmk] $L^p$-norm is indeed a norm for $p\geq 1$
    \end{itemize}
    \item[*]$L^\infty$ space
    \begin{itemize}
        \item For a measurable function $f:X\rightarrow \R\union \{\pm \infty\}$, the essential supremum norm is defined as \begin{equation*} \|f\|_\infty := \inf\{M>0 : \mu(\{|f|>M\})=0 \quad i.e.\quad |f|\leq M\;\, \mu-a.e.\}
        \end{equation*}
        \item $\LL^\infty :=\{f:X \rightarrow \R\union \{\pm \infty\} \,|\, \|f\|_\infty<\infty \}$ where we say such $f$ is essentially bounded. \; $\LL^\infty$ is a vector space. \; $L^\infty:=\LL^\infty/\sim$
    \end{itemize}
    \item Both H\"older's and Minkowski's inequality can be extended to the case of $p=\infty$
    \begin{itemize}
        \item (H\"older) If $f\in L^\infty\, ,\, g\in L^1$ then $\|fg\|_1\leq \|f\|_\infty\|g\|_1$
        \item (Minkowski) If $f,g\in L^\infty$ then $\|f+g\|_\infty\leq \|f\|_\infty+\|g\|_\infty$
    \end{itemize}
    \item[\rmk] $L^\infty$-norm (essential supremum norm) is indeed a norm on $L^\infty$
    \item $\|f\|_\infty = \inf\{\sup_{x\in X}|g(x)| : g\in \LL^\infty, \; f=g\;\,a.e.\}$
    \item Riesz-Fisher Theorem
    \begin{itemize}
        \item For $1\leq p\leq \infty$, $L^p$ is a Banach space.
    \end{itemize}   
    \item Approximation in $L^p$
    \begin{itemize}
        \item $1\leq p<\infty$. For any $f\in L^p$ and $\varepsilon>0$, $\exist$ a simple function $\phi\in L^p$ with $\|f-\phi\|_p<\varepsilon$
    \end{itemize}
    \item[\rmk] Why $L^p$ spaces ?
    \begin{itemize}
        \item Among all $L^p$ spaces, $p=1,2,\infty$ are the special ones.
        \item $L^2$ is a Hilbert space, which is easier to anlyze. $L^1$ is a natural one consisting of all integrable functions.
        \item One might ask whether we really need to consider $L^p$ spaces for $p\geq 1$ other than $p=1,2,\infty$. The answer is Yes.
        \begin{enumerate}
            \item In Fourier anlysis, there are many operators continuous on $L^p$ for $1<p<\infty$ but not on $L^1$ or $L^\infty$, such as the Hilbert transform.
            \item In probability theory, the most important distribution is the gaussian distribution  which belongs to $L^p$ for $1\leq p<\infty$. The second most importnat one could be the $p$-stable distribution for $0<p<2$ which belongs to $L^q$ for $q<p$ but not to $L^p$. This yields a heavy tailed process.
        \end{enumerate}
    \end{itemize}
\end{itemize}
\bigskip
\section{Signed Measures and The Radon-Nikodym Theorem}
\begin{itemize}
    \item[*] Signed measure
    \begin{itemize}
        \item $(X,\A)$ : a measurable space. A set function $\nu: \A\rightarrow \R\union \{\pm \infty\}$ is said to be a signed measure on $(X, \A)$ if \begin{enumerate}
            \item $\nu(\phi)=0$
            \item $\nu$ assumes at most one of the values $\pm \infty$.
            \item $\nu(\bigcup_{n}E_n)=\sum_n \nu(E_n)$ for any disjoint $\{E_n\}\seq \A$, where the RHS sum is absolutely convergent if LHS is finite.
            \item[\rmk] For the third condition, note that when the indices of $\{E_n\}_n$ change, LHS does not change. To prevent the change of RHS, the additional condition is added.
        \end{enumerate}
    \end{itemize}
    \item[\rmk] Every measure is a signed measure. For emphasizing the difference, we sometimes call a measure as a positive measure.
    \item $\mu_1,\, \mu_2$ : positive measures on $(X, \A)$. If at least one of them is finite measure, then $\nu=\mu_1-\mu_2$ is a signed measure. 
    \item[*] Extended $\mu$-integrable
    \begin{itemize}
        \item $f:X\rightarrow \R\union \{\pm \infty\}$ is said to be extended $\mu$-integrable if either one of $\int f^+\, d\mu$ or $\int f^-\, d\mu$ is finite. 
    \end{itemize}
    \item $(X, \A, \mu)$ : a measure space. If $f\in L^1(\mu)$ or $f$ is extended $\mu$-integrable, then $\nu : \A\rightarrow \R\union \{\pm \infty\}$ defined by $\nu(E):=\int_E f\, d\mu=\int_E f^+\, d\mu-\int_E f^-\,d\mu$ is a signed measure. \\(Signed measure can be understood as a generalization of function in this sense)
    \item A signed measure satiesfies continuity from above and from below.
    \item[*] Positive set, Negative set and Null set
    \begin{itemize}
        \item $\nu$ : a signed measure on $(X, \A)$. Let $E\in \A$
        \begin{enumerate}
            \item $E$ is a positive set (w.r.t. $\nu$) if $\nu(F)\geq 0 \; \forany F\subset E, F\in \A$. \quad Denote $E\geq_\nu 0$
            \item $E$ is a negative set (w.r.t. $\nu$) if $\nu(F)\leq 0 \; \forany F\subset E, F\in \A$ \quad Denote $E\leq_\nu 0$
            \item $E$ is a null set (w.r.t. $\nu$) if $\nu(F)= 0 \; \forany F\subset E, F\in \A$ \quad Denote $E=_\nu 0$
        \end{enumerate}
    \end{itemize}
    \item Elementary properties of positive sets
    \begin{enumerate}
        \item If $E\geq_\nu0$ then for any $F\subset E, F\in \A$, we have $F\geq_\nu0$
        \item If $E_n\geq_\nu0\; \forany n\in \N$ then $\bigcup_n E_n\geq_\nu 0$
    \end{enumerate} 
    \item Hahn decomposition thm
    \begin{itemize}
        \item If $\nu$ is a signed measure on $(X, \A)$, then $\exist P\geq_\nu0,\, N\leq_\nu0$ \;s.t. $X=P\union N$ is a partition. The choice of $(P, N)$ is unique up to null sets. 
        \item[\rmk] Uniqueness of Hahn decomposition upto null sets means that if $(P, N)$ and $(P', N')$ are two Hahn decompositions of $(X, \A, \nu)$ then $P\intersect N'=_\nu 0$ and $P'\intersect N=_\nu 0$
    \end{itemize}
    \item[*] Mutually singular signed measure
    \begin{itemize}
        \item Two signed measures $\mu$ and $\nu$ are said to be mutually signular if $ \exists$ a partition $X=E\union F$ with $E=_\mu 0$ and $F=_\nu 0$. Denote it as $\mu \perp \nu$
    \end{itemize} 
    \item Jordan decomposition thm
    \begin{itemize}
        \item If $\nu$ : a signed measure then, there are unique positive measures $\nu^+$ and $\nu^-$ s.t. $\nu=\nu^+-\nu^-$ and $\nu^+ \perp \nu^-$
        \item Given a Hahn decomposition $X=P\union N$ for $\nu$, \\$\nu^+$ and $\nu^-$ are given as $\nu^+(E)=\nu(E\intersect P)$ and $\nu^-(E)=\nu(E\intersect N)$
        \item[\rmk] Jordan decomposition is very similar to a unique decomposition of a measurable function $f=f^+-f^-$ where $f^+, f^-$ are both nonnegative and have disjoint supports.
    \end{itemize}
    \item Hahn decomposition and Jordan decomposition for  $\nu$ defined by $\nu(E)=\int_E f\, d\mu$
    \begin{itemize}
        \item For a measure space $(X, \A, \mu)$ and  (extended) $\mu$-integrable $f$, we have a signed measure $\nu$ defined by $\nu(E)=\int_E f\,d\mu$
        \item Hahn decomposition for $\nu$ is $X=P\union N$ with $P=\{f\geq0\}$ and $N=\{f<0\}$ \\( Other possible choice is $P=\{f>0\}$ and $N=\{f\leq 0\}$ )
        \item Jordan decomposition for $\nu$ is $\nu=\nu^+-\nu^-$ with \\$\nu^+(E)=\int_E f^+\, d\mu$ and $\nu^-(E)=\int_E f^-\,d\mu$
    \end{itemize}
    \item If $\mu$ is a positive measure and $\lambda_1, \lambda_2$ are signed measures on $(X, \A)$ with $\lambda_1 \perp \mu, \; \lambda_2\perp \mu$, then $(\lambda_1+\lambda_2)\perp \mu$
    \item[*] Total variation of signed measure \,\&  Finiteness of signed measure 
    \begin{itemize}
        \item For a signed measure $\nu$, the total variation of $\nu$ is defined by the positive measure $|\nu|=\nu^+ + \nu^-$. (This is similar to $|f|=f^+ + f^-$)
        \item A signed measure $\nu$ is said to be finite (or $\sigma$-finite) if $|\nu|$ is finite (or $\sigma$-finite)
    \end{itemize}
    \item If $\nu$ is a signed measure and $\mu$ is a positive measure then \begin{enumerate}
        \item $E=_\nu 0 \Leftrightarrow |\nu|(E)=0 \quad \forany E\in \A$
        \item $\nu\perp \mu \Leftrightarrow \nu^+\perp \mu, \, \nu^-\perp \mu\Leftrightarrow |\nu|\perp \mu$ 
    \end{enumerate}
    \item[*] Absolutely continuity of a signed measure w.r.t. a positive measure.
    \begin{itemize}
        \item $\nu$ is a signed measure and $\mu$ is a positive measure on $(X, \A)$. We say $\nu$ is absolutely continuous with respect to $\mu$ if $\mu(E)=0\Rightarrow \nu(E)=0\; \forany E\in \A$. Denote it as $\nu \ll \mu$
    \end{itemize}
    \item If $(X, \A, \mu)$ is a measure space and $f$ is extended $\mu$-integrable function, then the signed measure $\nu$ defined by $\nu(E)=\int_E f\, d\mu$ is absolutely continuous w.r.t. $\mu$
    \item Lebesgue decomposition
    \begin{itemize}
        \item If $\mu$ is a $\sigma$-finite positive measure on $(X, \A)$, then a $\sigma$-finite signed measure $\nu$ on $(X, \A)$ is uniquely decomposed as $\nu=\nu_1+\nu_2$ with $\nu_1\ll \mu$ and $\nu_2 \perp \mu$ where $\nu_1, \nu_2$ are signed measures. 
        \item Moreover, $\exists$ a $\mu$-null set $B\in \A$\, s.t. $\nu_1(E)=\nu(E\diff B), \;\nu_2(E)=\nu(E\intersect B)\; \forany E\in \A$ 
        \item[\rmk] This decomposition tells us that absolute continuity is, in a sense, ``opposite'' to mutual singularity. 
    \end{itemize}
    \item Radon-Nikodym thm
    \begin{itemize}
        \item If $\mu$ is a $\sigma$-finite positive measure and $\nu$ is a $\sigma$-finite signed measure on $(X, \A)$ s.t. $\nu\ll\mu$, then $\exists$ a unique extended $\mu$-integrable function $g:X\rightarrow \R$ satisfying $\nu(E)=\int_E g\, d\mu \; \forany E\in \A$
        \item The function $g$ is called as the Radon-Nikodym derivative of $\nu$ w.r.t. $\mu$.  Denote as  $$d\nu= g \, d\mu ,\quad g=\frac{d\nu}{d\mu}$$
    \end{itemize}
    \item Conditional expectation
    \begin{itemize}
        \item $(X, \A, \mu)$ : a finite measure space. $\B$ : a sub $\sigma$-algebra of $\A$. Let $\nu=\mu|_\B$. \\Then for any $f\in L^1(X, \A, \mu)$, \;$\exist g\in L^1(X, \B, \nu)$ \, s.t. $\int_E f\, d\mu=\int_E g\, d\nu \quad \forany E\in \B$ 
        \item[\rmk] For $\B\subset \A$, \;`$f$ is $\A$-measurable' $\nRightarrow$ `$f$ is $\B$-measurable'. Thus we should find \\$\B$- measurable $g$ satisfying $\int_E f\, d\mu=\int_E g\, d\nu \quad \forany E\in \B$
        \item Such $g$ is unique in the sense that if $\exists$ another such function $g'$, then $g=g'\; \nu-a.e.$ 
        \item In probability theory, $g$ is called as the conditional expectation of $f$. \\ Denote it as $g=E\,[\,f\,|\,\B\,]$
    \end{itemize}
\end{itemize}



\end{document}
